%/export/xray11/sdc1g08/HDData/Documents/blanklatexfiles/poster/ngc1365

% packages and set-up

\documentclass[final]{beamer}
\usepackage{grffile}
\mode<presentation>{\usetheme{I6pd2}}
\usepackage[english]{babel}
\usepackage[latin1]{inputenc}
\usepackage{multirow}
\usepackage{amsmath,amsthm, amssymb, latexsym}
\usepackage[orientation=portrait,size=a0,scale=1.4,debug]{beamerposter}
\usepackage{caption}
\captionsetup{labelformat=empty}
\usepackage{wrapfig}
\usepackage{array,booktabs,tabularx}
\newcolumntype{Z}{>{\centering\arraybackslash}X} % centered tabularx columns
\newcommand{\pphantom}{\textcolor{ta3aluminium}} % phantom introduces a vertical space 
						  % in p formatted table columns??!!
\usepackage{hhline}\fboxrule=4pt%border thickness

\listfiles

%%%%%%%%%%%%%%%%%%%%%%%%%%%%%%%%%%%%%%%%%%%%%%%%%%%%%%%%%%%%%%%%%%%%%%%%%%%%%%%%%%%%%%

\graphicspath{{figures/}}
 
\title{ The Long Term X-Ray Spectral Variability of\\ \vspace{1cm}
  \hspace{0.1cm} NGC1365 with SWIFT \vspace{1cm}}
  
\author{Sam Connolly $ ^{\star}$, Ian McHardy}
\institute{{\small $^{\star}$ Email: sdc1g08@soton.ac.uk}\hspace{370pt}
	      School of Physics and Astronomy, University of Southampton, UK}
\date[20th May 2013]{20th May 2013}
\subtitle{\hspace{18cm}}

%%%%%%%%%%%%%%%%%%%%%%%%%%%%%%%%%%%%%%%%%%%%%%%%%%%%%%%%%%%%%%%%%%%%%%%%%%%%%%%%%%%%%%
\newlength{\columnheight}
\setlength{\columnheight}{96cm}


%%%%%%%%%%%%%%%%%%%%%%%%%%%%%%%%%%%%%%%%%%%%%%%%%%%%%%%%%%%%%%%%%%%%%%%%%%%%%%%%%%%%%%
\begin{document}

\begin{frame}

            \begin{block}{\begin{center} Abstract \end{center}} 
			We present long-term spectral variability in SWIFT data of the Seyfert galaxy NGC1365. 
			The data cover both a large time period (six years) and a very wide flux range.
			The spectra have been fitted using a variety of models, in order to 
			discover the nature of and reasons for the observed variation. It has 
			been found that variation in the degree of absorption is the most 
			likely cause of spectral changes. The primary cause of the variation
			in absorption is found to be most likely due to changes in the
			column density of the absorbing material; changes in the ionisation state of the
			absorbing material alone are found to be insufficient to account for the spectral changes observed. 
			Furthermore, spectral fits seem to show that the degree of absorption decreases with 
			increasing source flux, implying varying obscuration of the source.
            \end{block}
            
            \vfill

  \begin{columns}
  
%================================================================================
% -------------------- COLUMN 1 -------------------------------------------------
%================================================================================
    
    \begin{column}{.5\textwidth}
      \begin{beamercolorbox}[center,wd=\textwidth]{postercolumn}
        \begin{minipage}[T]{.98\textwidth}  % tweaks the width, makes a new \textwidth
								
          \parbox[t][\columnheight]{\textwidth}{ 

            % ----------------- BOX 1 -----------------------------------------------

            \begin{block}{1. Introduction}
            
			
 		The Seyfert galaxy NGC1365 is known to be a highly variable X-ray source [1]. Previous observations have seen large
 		changes in the column density of absorbing material, causing significant spectral variation on time scales of weeks to years [2]. 
 		In this work, we use six years of SWIFT data to study long term trends in the spectral variability of NGC1365 and to investigate the relationship 
 		between X-ray flux and the observed X-ray spectrum.
		
            \end{block}

            \vfill
            
	    \vspace{-120pt}
	    
	    % ----------------- BOX 2 ------------------------------------------

            \begin{block}{2. Spectral Hardness} 
            
		\begin{figure}
			\hspace{-10pt}
			\fcolorbox{gray}{white}{
			\includegraphics[width=0.49\linewidth,height=0.4\linewidth,clip = true, trim = 0 0 0 0]{binhardnessvhard.eps}
			\includegraphics[width=0.49\linewidth,height=0.4\linewidth,clip = true, trim = 0 0 0 0]{binhardvsoft.eps}
			}
			\caption{Left: Hard flux (x) against 
			hardness ratio (y). Right: Hard flux (x) against soft flux. (Data binned)}
		\end{figure}
				
		The hardness ratio shows the spectrum to be extremely soft at low fluxes. 
		The small amount of scatter implies that the shape of the spectrum can be 
		assumed to be similar at a given flux level independent of time. This allowed spectra observed 
		at a similar flux, but in different epochs, to be combined; 52 spectra were 
		binned according to the X-ray flux at the time of observation and 
		combined, resulting in a set of spectra representing the spectral shape at each flux level.
		
		
		
			
            \end{block}

            \vfill    
	    \vspace{-120pt}  
          
	
	                
             % ----------------- BOX 3 ---------------------------------------------------


           \begin{block}{3. Spectral Modelling} 
           
           %\hspace{-20pt}
           
           \begin{wrapfigure}{L}{0.65\textwidth}
	      \begin{table}[ht!]
		\label{table:second}
		\centering
		\footnotesize
		\begin{tabular}{| p{7.5cm}|| p{3cm} | p{3cm} | p{5.2cm} || p{1.5cm} | p{1.7cm}  |} \hline
		
		
		
		\multirow{2}{*}{Model}		   	& \multicolumn{3}{|c||}{Parameters}			&\multirow{2}{*}{$\chi^2_{Red}$}&\multirow{2}{*}{DoF} \\ \hhline{~---~~}   
							& Fixed		& Tied		& Free			& 		& \\ \hline
		Absorbed and  			   	& spectral 	& - 		& column density, 	& 1.56 		& 1077 \\ 
		unabsorbed power laws		    	& index		&		& ionisation		&		&      \\ \hline			  
		Absorbed and 				& spectral 	& ionisation 	& column density 	& 1.56		& 1087 \\ 
		unabsorbed power laws			& index		&		&			&		&      \\ \hline
		Absorbed and 				& spectral	& column  	& ionisation  		& 2.61		& 1087 \\ 
		unabsorbed power laws			& index		& density	&			&		&      \\ \hline
		Single, absorbed power					& -		&  		& column density,	& 1.52		& 1077 \\ 
		law 				&		&		& spectral index,	&		&	\\
							&	 	& 		& ionisation		&		&      \\ \hline
		Absorbed and  				& -		& column 	& spectral index, 	& 2.20		& 1077 \\ 
		unabsorbed power laws			& 		& density	& ionisation		&		&      \\ \hline
		
		
		
		\end{tabular}
		
		\caption{$\chi^2_{Red}$ values and no. of degrees of freedom (DoF) of five models fitted to the average spectra.}
		
		\end{table}
		\end{wrapfigure}
		
              	Spectral fits show that the model which best describes the data consists of two power laws, 
              	one unabsorbed and one undergoing absorption with a varying column density.
              	In each case, the spectral index was assumed to be constant, at the value found
              	by Risaliti et. al (2013) [3].
              	             	
		
            \end{block}
          
	   
            
                    % ----------------- BOX 4 -----------------------------------------------

            \begin{block}{4. Two-Component Spectral Variability}
            
		\vspace{-10pt}		
		\begin{wrapfigure}{L}{0.66\textwidth}
			\vspace{-10pt}
			\fcolorbox{gray}{white}{
			\hspace{-30pt}
			\includegraphics[width=.65\linewidth,height=.99\linewidth,
			angle = 270,trim = 30 100 10 70,clip = true]{spectra.eps}\hspace{1pt}}
			\caption{Sample of the average spectra produced by combining 
			spectra in the same flux range.}
			\vspace{-30pt}
		\end{wrapfigure} 
		
		Here we show plots from the best fitting model, described above.
		In this case, the spectrum is composed of a weakly varying unabsorbed component
		and a strongly varying absorbed component, whose absorption varies
		inversely with luminosity. The spectral variability observed cannot
		be accounted for by changes in ionisation, requiring a change in the
		amount of absorbing material between the observer and the X-ray source.
 		
		
            \end{block}

            \vfill
            
	    \vspace{-40pt}
     




		


			} % end of parbox!

	% --- end column ---

        \end{minipage}
      \end{beamercolorbox}
    \end{column}

%================================================================================
% ----------- RIGHT COLUMN ------------------------------------------------------
%================================================================================


    \begin{column}{.5\textwidth}
      \begin{beamercolorbox}[center,wd=\textwidth]{postercolumn}
        \begin{minipage}[T]{.97\textwidth} % tweaks the width, makes a new \textwidth
          \parbox[t][\columnheight]{\textwidth}{

          \vspace{20pt}

          
 
         %---- Box 5 ------------------------------------------------

	 \vspace{-20pt}
	    
         \begin{block}{5. A Possible Link Between Source Flux and Column Density} 
	
		The best-fitting model requires the normalisations of the power
              	laws to vary; these are approximately correlated, as would be 
              	expected if the emission were from the same source. In this case,
              	the two components could correspond to direct, absorbed emission
              	and unabsorbed emission which is direct and/or scattered. 
	
		\begin{figure}
			\fcolorbox{gray}{white}{
			\includegraphics[width=0.47\linewidth,height=0.43\linewidth,trim = 20 0 50 0, clip=true]{nHvNorm2.eps}\hspace{1pt}
			\includegraphics[width=0.47\linewidth,height=0.43\linewidth,trim = 20 0 10 0, clip=true]{norm2Vnorm1.eps}\hspace{1pt}
			}
			\caption{The normalisation parameter of the absorbed power law 
				(x) against the column density of the absorbing material (y) in
				the model described above.}
		\end{figure}  
				
		A possible link between the normalisation of the absorbed power law and the column density of the aborbing matter
		is revealed by plotting thise two paramaters, showing a decreasing column density with increasing source flux.
		
		
		
		\vspace{20pt} As models involving varying ionisation do not fit the data well, it seems that
		there is an actual change in the amount of material between the observer and the source.
		In the AGN wind model proposed by Elvis (2000) [4], absorbing material arises from a narrow
		range of disc radii. A higher accretion rate in this model leads to both a higher disk temperature and a greater
		X-ray luminosity, causing the wind to arise at larger radii. It is possible that this
		could lead to less obscuration of the X-ray source, giving a physical mechanism for the oberved decrease in column
		density with increased flux.

        \end{block}            

	        
	  \vspace{30pt}  
         
         
         
	 %---- Box 6 ------------------------------------------------
	
         \vspace{10pt}

         \begin{block}{6. Modelling Hardness Variation with Flux}   
            
		\begin{figure}
			\fcolorbox{gray}{white}{
			\includegraphics[height=0.3\linewidth,width=0.97\linewidth,trim = 0 0 0 0, clip=true]{hardnessModelling.eps}
			}
			\caption{Top: A sample of many simulated spectra with a
				varying absorbing column density. Middle: The 
				corresponding hard flux (x) against hardness (y) plot. 
				Bottom: The corresponding hard flux (x) against soft 
				flux (y) plot.}
  			\end{figure}
  				
		Artifical spectra, created using each model,
		show that the hardness ratios seen in the data can only be accurately
		reproduced by the model described above. 
				
         \end{block}  
          
         \vspace{40pt}   


	%---- Box 7 ------------------------------------------------      

	\begin{block}{7. Conclusions}
				
		The SWIFT spectra of NGC1365 show large variation, the cause of which
		is found to be changes in the column density of absorbing material.
		Furthermore, the column density appears to decrease with increasing X-ray flux. 
		
		
	\end{block}

         \vspace{40pt} 
        %References

		  \textcolor{yellow}{\tiny{\vspace{-20pt} References: (1) Risaliti, G., Miniutti, G., Evlis, M., Fabbiano, G., Salvati, M., Baldi, A., Braito, V., 
		  Bianchi, S., Matt, G., Reeves, J., Soria, R., Zezas, A., \\ \vspace{-20pt} 2009, ApJ,696,160. (2) Brenneman, L.W., Risaliti, G., Evlis, M., Nardini, E., 2013 MNRAS, 
		  429, 2662.  (3) Risaliti, G., Harrison, F.A., Madsen, K.K., Walton, D.J.,\\ \vspace{-20pt} Boggs, S.E., Christensen,  F.E., Craig, W.W., Grefenstette, B.W., 
		  Hailey, C.J., Nardini, E., Stern, D., Zhang, W.W., 1013, Nature, 494, 449. (4) Elvis, M., 2000, \\ \vspace{-20pt} ApJ, 545, 63.}}
		 
		  \vfill		

      
	} % end of par box!  

          

    % ---------------------------------------------------------%
    % end the column

        \end{minipage}
      \end{beamercolorbox}
    \end{column}
    % ---------------------------------------------------------%

    % end the columns
  \end{columns}

\end{frame}

\end{document}

%%%%%%%%%%%%%%%%%%%%%%%%%%%%%%%%%%%%%%%%%%%%%%%%%%%%%%%%%%%%%%%%%%%%%%%%%%%%%%%%%%%%%%%%%%%%%%%%%%%%

